\problem{}
Given a function rand2() that returns 0 or 1 with equal probability, implement rand3() using rand2() that returns 0, 1 or 2 with equal probability. Minimize the number of calls to rand2() method. Prove the correctness.
\solution{
\begin{algorithm}[h]
	\caption{Rand3()} %算法的名字
	\hspace*{0.02in} {\bf Input:} %算法的输入, \hspace*{0.02in}用来控制位置,同时利用 \\ 进行换行
	rand2()\\
	\hspace*{0.02in} {\bf Output:} %算法的结果输出
	rand3()
	\begin{algorithmic}[1]
		\State int x,y; % \State 后写一般语句
		\While{x==1 and y==0} % While语句,需要和EndWhile对应
		\State x = rand2()
		\State y = rand2()
		\EndWhile
		\State \Return x+y
	\end{algorithmic}
\end{algorithm}

\textbf{Proof:} To prove that our algorithm returns 0,1,2 with equal probability, we start to discuss about the case $p(RAND3()==1)$, because there is no case that $x==1$ and $y==0$ at the same time. It can be inferred that:
$$p(RAND3()=1)=p(x+y=1)=\frac{p(x=0,y=1)}{p(x=0,y=0)+p(x=0,y=1)+p(x=1,y=1)}$$
For $p(x=0,y=0)=p(x=0,y=1)=p(x=1,y=1)=\frac{1}{4}$, $p(RAND3()=1)=\frac{1}{3}$.

Evidenced by the same token, $p(RAND3()=0)=p(RAND3()=2)=\frac{1}{3}$.
}