\problem{}
Suppose there is Super Mario: the Game of eating Gold coins.
The goal of this game is to get as many gold coins as possible under the rules.
\par
Now let's introduce the game settings.
\par
There are N × N grids as game map, in which there are only three categories for each grid, which are represented by values 0,1,2:

1. The grid value is 0: this is an open space, which can be directly passed.

2. The grid value is 1: there is a gold coin in the open space, which can be passed and the number of gold coins obtained increases by one.

3. The grid value is 2: this is an obstacle, unable to pass.

In the game, Mario needs to do these things:
\par
Mario starts from grid position (0, 0) and arrives at (N-1, N-1). In the game, Mario can only walk down or right, and can only walk through the effective grid ( the grid value is 0 or 1).

When Mario arrives at (N-1, N-1), Mario should continue to walk back to (0, 0). At this stage, Mario can only walk up or left, and can only cross the effective grid ( the grid value is 0 or 1).

When Mario goes from (0,0) to (N-1, N-1) and then from (N-1,N-1) to (0,0), Mario finishes the game and counts the gold coins obtained.
If there is no path that Mario can pass between (0, 0) and (N-1, N-1), the game ends and the number of gold coins obtained is 0.

NOTE: when Mario passes a grid and there is a gold coin in the grid (the grid value is 1), Mario will take the gold coin and the grid will be empty (the grid value becomes 0).

Design an effective algorithm to play the game and return the maximum gold coins which can be obtained. Analyze the algorithm’s time complexity and space complexity.

Example:
$$
Input: \text{grids} = \begin{bmatrix}
  0&  1&  1& 2\\
  1&  0&  2& 0\\
  0 & 1 & 1 & 1\\
  0& 2&  1  &0
\end{bmatrix}, N=4
$$
$$
output: \text{ max number of gold coins} = 6
$$

\textbf{Solution:}

Use the dynamic programming to solve the problem. There is a matrix $dp[n,n]$ to save the result. The elements in $dp$ symbolize the maximum gold coins can be obtained. Because Mario goes from (0,0) to (N-1,N-1) and then from (N-1,N-1) to (0,0), there will be two times dynamic programming in the algorithm, and \textbf{if the coin is taken from (0,0) to (N-1,N-1), the grid value must set to zero}. The first state transition equation is as following($dp[i,j]=-1$ symbolize there is no way to grid[i,j] or here is an obstacle):
\begin{equation}
	\label{eq6}
	dp_1[i,j]=\left\{
	\begin{aligned}
		\max(dp_1[i-1,j],dp[i,j-1]) + grid[i,j] & , & dp_1[i-1,j] \ne -1\wedge grid[i,j]\ne 2 \\
		\max(dp_1[i-1,j],dp[i,j-1]) + grid[i,j] & , & dp_1[i,j-1] \ne -1\wedge grid[i,j]\ne 2 \\
		-1 & , & else.
	\end{aligned}
	\right.
\end{equation}

The second state transition equation is as following:
\begin{equation}
	\label{eq6}
	dp_2[i,j]=\left\{
	\begin{aligned}
		\max(dp_2[i+1,j],dp[i,j+1]) + grid[i,j] & , & dp_2[i+1,j] \ne -1\wedge grid[i,j]\ne 2 \\
		\max(dp_2[i+1,j],dp[i,j+1]) + grid[i,j] & , & dp_2[i,j+1] \ne -1\wedge grid[i,j]\ne 2 \\
		-1 & , & else.
	\end{aligned}
	\right.
\end{equation}
$$Coin_{max} = max(dp_1[n,n],0)+max(dp_2[0,0],0)$$

\textbf{Analysis:}

The computation of subproblem cost $O(1)$ and there are 2*n*n subproblems, so the time complexity is $O(n^2)$. Since we make two n*n matrix, so the space complexity is $O(n^2)$.
